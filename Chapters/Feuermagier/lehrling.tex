\chapter{Wahre Magie}
Am n�chsten Abend setzt uns \Kajira\ in der N�he einer Stadt ab, und wir
schaffen es noch rechtzeitig hinein, bevor die Tore f�r die Nacht geschlossen
werden. Von den letzten paar Talern in meiner Tasche miete ich uns ein Zimmer
f�r die Nacht -- endlich wieder ein Bett. Irgendwo hat \Yasmin\ auch noch ein
paar Fr�chte aufgetrieben, was gut ist, denn f�r eine Mahlzeit reicht mein Geld
nicht mehr. Wir m�ssen dringend wieder etwas Arbeit finden.

Schweigend nehmen wir unser karges Mahl zu uns, und gleich nachdem \Yasmin\ aus
dem Bad am Ende des Flurs zur�ckgekommen ist, gehe auch ich mich f�r die Nacht
fertig machen. Zu meiner freudigen �berraschung gibt es eine Art Dusche, ein
Eimer mit einem Seil, der �ber einer mit Steinen gefliesten Wanne h�ngt. Auch
wenn das Wasser selbstverst�ndlich kalt ist, genie�e ich die W�sche in vollen
Z�gen. Sogar Seife gibt es hier. Es ist zwar anschlie�end um so unangenehmer, in
die schmutzige und durchgeschwitzte Kleidung zu steigen, trotzdem tat das nach
den letzten Wochen mal wieder gut.

Bereits vor Sonnenaufgang weckt mich \Yasmin. Dabei hatte ich gehofft, endlich
mal wieder richtig ausschlafen zu k�nnen. Aber sie meint, sie h�tte gestern
Abend eine Markthalle gesehen und hofft, dass wir dort heute morgen noch ein
wenig Geld verdienen k�nnten, bevor wir weiterziehen. Warum will sie gleich
weiterziehen? Wir sind doch gerade erst angekommen! Doch noch bevor ich die
Frage aussprechen kann, ist sie zur T�r hinaus. Ich schnappe mir meine Sachen
und den Apfel, den ich gestern vom Abendbrot �brig behalten habe, dann eile ich
ihr nach.

Wir haben Gl�ck. In der nur langsam erwachenden Stadt k�nnen wir uns tats�chlich
ein paar M�nzen verdienen, indem wir f�r die H�ndler Kisten von Wagen laden und
quer durch die Markthalle schleppen. Im Gegensatz zu den Stra�en drau�en geht es
hier drinnen schon zu so fr�her Stunde zu, wie in einem Bienenstock. Irgendwann
winkt mich \Yasmin\ nach drau�en, die Sonne steht bereits ein gutes St�ck �ber
den Mauern der Stadt. "`Das soll reichen, mehr k�nnen wir hier nicht
verdienen,"' sagt sie. Dann z�hlen wir unser Geld, teilen es zwischen uns auf
und suchen dann einen Weg aus der Stadt hinaus.

Wir finden ein Tor im Norden der Stadt, fast genau gegen�ber dem, durch das wir
hereingekommen sind. Doch die chaotische Bebauung innerhalb der Stadtmauern
macht es uns nicht leicht, den Weg dorthin zu finden. Immerwieder k�nnen wir es
zwischen zwei H�usern sehen, doch wie in einem Labyrinth scheinen alle direkten
Wege in Sackgassen zu enden. So kommt es, dass ich erst ein gutes St�ck vor der
Stadt, nachdem auch die Au�ensiedlungen schon hinter uns liegen, dazu komme,
\Yasmin\ nach unserem hastigen Aufbruch zu fragen.

Sie bleibt so abrupt stehen, dass ich bereits zwei Schritte vor ihr bin, bis ich
selbst anhalten kann. "`\Hendrik, was f�r eine Frage ist das? Hast Du vergessen,
dass wir verfolgt werden?"' Ich sehe sie irritiert an. "`Ich dachte, unsere
Verfolger w�ren auf der Jagd nach Drachen gewesen, und wir nur zuf�llig in ihrem
Weg?"' -- "`Wenn jemand Drachen jagen will, denkst Du, er nimmt dann winzige
Giftpfeile mit?"' Nein, wohl eher nicht. "`Es war vielleicht ein Zufall, dass
wir in den Angriff auf die Drachen hineingeplatzt sind, aber dass wir verfolgt
wurden, war sicher kein Zufall."' Und damit l�sst sie mich stehen und folgt
wieder z�gigen Schrittes der Stra�e.
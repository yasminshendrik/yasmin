\section{Magie erkl�rt}
"`Das was in dieser Welt mit den vier magischen Elementen Wasser, Erde, Feuer
und Luft erkl�rt wird, sind nach dem Verst�ndnis Eurer Welt nur verschiedene
Aspekte einer einzigen physikalischen Gr��e: der Energie. Magie ist im Grunde
also nichts anderes als das Manipulieren der Energie, die uns umgibt. Nur dass
wir in dieser Welt einen, ich sage mal, direkteren Zugang zu dieser Energie
haben. Feuer, Dein Lieblingselement, manipuliert die Temperatur der Dinge. Sein
vorgebliches Gegenst�ck, das Wasser, beeinflusst die Bewegung Dinge, was mit
Fl�ssigkeiten am leichtesten zu funktionieren scheint. Deutlich schwerer als die
beiden ist der Umgang mit Luftmagie, denn sie arbeitet mit Hilfe von Druck und
Dichte. So setzt sie zum Beispiel Gase in Bewegung, indem sie Druckunterschiede
erzeugt. Die Schilde der Luftmagier werden erzeugt, indem auf eng begrenztem
Raum die Dichte der Luft gewaltig erh�ht wird. Noch komlizierter ist die
Erdmagie, denn sie basiert auf �nderungen der Lageenergie von Gegenst�nden, am
besten klappt das mit festen K�rpern. Entzieht man einem K�rper Lageenergie, so
steigt er auf, gibt man ihm welche, sinkt er hinab. Bis zu einem gewissen Grad
kann man somit Steine fliegen lassen, wenn man so will.

Zu den urspr�nglichen vier Elementen kamen durch uns noch zwei weitere hinzu:
Licht und Schatten. So hat man sie jedenfalls benannt, weil man der Ansicht war,
dass Magie immer paarweise existiert. Unsere Magie wurde als die des Lichtes
bekannt, ihr vermeintliches Gegenst�ck musste also Schattenmagie sein. Wie
unsere eigene Magie funktioniert, wissen wir inzwischen recht gut: wir �ndern
die Eigenschaften von Lichtwellen. Es m�sste eigentlich auch mit anderen Wellen
funktionieren, aber soweit ich wei�, hat das bisher noch niemand ernsthaft
ausprobiert. Worauf allerdings die Schattenmagie basiert, ist noch v�llig
unklar. Es gibt zwar etliche Theorien, eine absurder als die andere, aber keine
davon konnte auch nur ansatzweise bewiesen werden.

Was dieses Modell mit seinen magischen Paaren v�llig au�er Acht l�sst und sich
auch nicht vern�nftig mit Eurer Wissenschaft erkl�ren l�sst, sind die magischen
F�higkeiten einiger weniger bekannter V�lker, wie zum Beispiel der Feen. In
manchen Regionen k�nnte man sogar auf dem Scheiterhaufen landen, w�rde man
behaupten, dass neben den sechs - oder vier, je nach Ansicht der vorherrschenden
Relegion - magischen Elementen noch weitere existieren k�nnten. Oder gar eine
v�llig andere Erkl�rung der Magie.

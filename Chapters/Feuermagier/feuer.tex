\section{Feuer lernen}
Ich �ffne meine Hand, mit einem Ruck, genau, wie sie es mir gezeigt hat. Nicht,
dass ich wirklich daran glauben w�rde. Doch erschrocken sehe ich meine Hand an,
ein kurzes Aufflackern, dann Rauch. Tats�chlich, Feuer aus meiner Hand! Aus
meiner! Ich sehe sie an, ihr Gesicht eine Mischung aus Belus�tigung und Stolz,
ein fr�hliches L�cheln auf ihren Lippen. Sie nimmt meine Hand und schlie�t sie
erneut zu einer Faust. Mit der anderen Hand f�hrt sie mir �ber die Augen, ich
verstehe die Geste und schlie�e sie.

Pl�tzlich ist da diese Stimme im meinem Kopf, die laut \magic{Feuer} sagt.
Verwirrt rei�e ich die Augen auf und starre sie an. Doch sie l�chelt nur und
nickt, die Augen auf meine Hand gerichtet. Ich folge ihrem Blick und stelle zu
meinem Erstaunen fest, dass meine Hand ge�ffnet ist. Und noch viel mehr erstaunt
mich die kleine Flamme, die gleichm��ig dicht �ber meine Handfl�che brennt.

Der Anblick l�sst mich geradezu erstarren, ich traue mich nicht, meine Hand auch
nur einen Millimeter zu bewegen. Mein Blick wandert wieder zur�ck zu ihr, stolz
sieht sie mir in die Augen. Ich kann es kaum glauben, Feuer, aus meiner eigenen
Hand, es steckt tats�chlich Magie in mir! Doch war ich das wirklich selbst? Oder
wirkt nur ihre Magie, ich wei� nicht, durch mich hindurch, sozusagen? In meine
Zweifel mischt sich pl�tzlich ein anderes Gef�hl. Hei�! Ich hatte das Feuer ganz
vergessen. Vor Schreck sch�ttele ich meine Hand, als wollte ich ein Streichholz
l�schen. Und tats�chlich, das Feuer erlischt, doch es hinterl�sst eine
verbrannte Stelle in meiner Haut.

Doch dann geschieht etwas Unerwartetes: zum ersten Mal h�re ich sie laut und
herzhaft lachen, ein fr�hliches, helles Lachen. Dann nimmt sie meine verbrannte
Hand zwischen ihre H�nde. Unsere Blicke treffen sich, und in diesem Moment sp�re
ich eine angenehme K�hlung auf der verkohlten Haut, als w�rde ganz weiches
Wasser dar�ber laufen. Doch gleichzeitig ist da auch eine W�rme, die aus dem
inneren meiner Hand zu kommen scheint. Und noch w�hrend ich versuche, meine
Wahrnehmungen einzuordnen, l�sst sie mich los.

Schon wieder gibt es etwas zum Staunen f�r mich, denn so genau ich auch meine
Hand untersuche, die Brandwunde ist restlos verschwunden.
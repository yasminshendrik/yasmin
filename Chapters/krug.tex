\section{Krug}
Ganz allein sitze ich an meinem Tisch. Zwar sind die Tische durch Regale und
andere Einbauten gut vor dem Einblick anderer G�ste gesch�tzt, doch jetzt, da
das Lokal noch so gut wie leer ist, f�hle ich mich geradezu einsam. In meinem
R�cken sitzen zwei G�ste hinter dem Sichtschutz und da es ansonsten fast
totenstill ist, h�re ich unfreiwillig ihrem zwar leise, aber dennoch gut h�rbar
gef�hrten Gespr�ch vor. Worum es geht erschlie�t sich mir allerdings nicht. Ich
kann nichteinmal herausfinden, in welchem Verh�ltnis die beiden zueinander
stehen. Sie wirken zwar sehr vertraut in ihrem Umgang miteinander, so wie
Freunde seit Kindestagen. Doch zugleich hat es den Anschein, als wenn sie sich
noch nie zuvor gesehen haben. Verwirrend, und so versuche ich ihre Stimmen zu
ignorieren.

Der Kellner, ein �lterer Herr mit dem Auftreten eines Butlers, n�hert sich
meiner Nische. Ich bestelle etwas zu trinken und das aktuelle Tagesmen�. Was ein
Unterschied zu den Schenken und Gasth�fen, in denen ich sonst so verkehre. Nicht
die Einrichtung oder die Speisen -- auch wenn letztere hier irritierend
komplizierte Namen haben und mit Zutaten zubereitet werden, von denen ich noch
nie geh�rt habe -- sondern wie das Personal gegen�ber den G�sten auftritt ist
\ldots\ ungewohnt. Angefangen von der netten Empfangsdame, die mich an meinen
Tisch gef�hrt hat, �ber die vielen Bediensteten, die, obwohl mit Tellern schwer
beladen, freundlich l�chelnd dem Gast den Vortritt lassen bis hin zu eben diesem
Kellner, der bei n�herer Betrachtung so was �hnliches wie der Chef des Lokals zu
sein scheint. Sie alle strahlen eins aus: hier verkehren normalerweise reichere
Leute, die sich keinen unh�flichen, ungepflegten Wirt gefallen lie�en, wie man
ihn in den Restaurants des normalen Volks nur allzu oft antrifft.

Der Sturm drau�en hat nachgelassen, jetzt regnet es nur noch gleichm��ig in
diesem tr�b-grauen Sp�tnachmittag. Schweigend sehe ich den Regentropfen zu, wie
sie auf das Pflaster der Stra�e prasseln. Ab und zu hastet eine einsame Gestalt
mit eng um sich geschlungenem Mantel und hochgezogenen Schultern vorbei. Sonst
ist es drau�en so unbelebt wie hier drinnen. Nur noch viel ungem�tlicher. Mein
eigener Mantel h�ngt triefnass in der Gaderobe. Ein ungewohntes Gef�hl, fast als
w�re ich nackt. Wehe, irgendwas aus meinen Taschen geht verloren. Und so f�hle
ich mich trotz der eigentlich ganz angenehmen Umgebung und der noch viel
angenehmeren W�rme des Kamins an der Wand extrem unwohl.
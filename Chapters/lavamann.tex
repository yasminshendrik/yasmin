\section{Der Mann aus Lava}
Vorsichtig bl�ttere ich eine weitere Seite des br�chigen Pergaments um. Auch
wenn ich nichts von dem lesen kann, was in enger Schrift auf den Seiten
geschrieben steht, fasziniert mich dieses Werk. Und es  hat jede Menge fein gezeichneter
Bilder und Skizzen, die ich mir stundenlang ansehen k�nnte. Alle m�glichen
Pflanzen sind dort dargestellt, fremdartige Landschaften, seltsame Wesen, wobei
ich mir manchmal nicht sicher bin, ob ich nun Pflanze, Tier oder
Landschaft vor mir habe. Das eine oder andere scheint auch eine technische Skizze von
irgendwelchen seltsamen Apparaten zu sein. Oder irgendwelchen Vorg�ngen. Doch um
zu verstehen, was ich hier sehe, m�sste ich die Texte dazu lesen k�nnen. Ich
sehe von dem Buch auf, lasse meinen Blick durch den Gastraum schweifen. Nein,
hier finde ich niemanden, der mir das hier vorlesen k�nnte. Und dabei ist das
hier noch eines der besseren Gasth�user meiner bisherigen Reise.

Gerade will ich mich wieder meinem Buch zuwenden, als eine Gestalt hereinkommt,
die ich selbst in diesem Buch noch nicht gesehen habe. Und kaum, dass die T�r
hinter ihr in den Rahmen geschlagen ist, wird es totstill im Raum. Unheimlich
laut knistert das Feuer im Kamin. Unbeirrt und zielstrebig bewegt sie sich auf
den Kamin zu, wo mir jetzt zum ersten Mal eine Handvoll kunstvoll behauene
Steinquader auffallen, die dort in loser Ordnung verstreut sind. Etwas an der
Art und Weise, wie sich diese Figur bewegt, irritiert mich. Nat�rlich ist da
einerseits die Tatsache, dass auf ihrem Rumpf ein Kopf fehlt und au�erdem
sind da auch noch ihre Arme, die im Umfang ihren Beinen gleichen und fast bis
zum Boden reichen. Aber diese Art, wie sich der K�rper auf den zwei Beinen
vorw�rts w�lzt -- mir f�llt kein anderes Wort ein, um das zu beschreiben -- ist
ein solch ungewohnter Anblick, dass es mir schwer f�llt, nicht hinzusehen. Doch
ich zwinge mich, den Blick von diesem Wesen zu wenden und erneut durch das Lokal
schweifen zu lassen. Selbst der Wirt und seine Kellnerinnen sind wie erstarrt,
ihre Augen folgen gebannt den Bewegungen des Neuank�mmlings.

Dieser wiederum scheint die Aufmerksamkeit, die ihm hier zuteil wird, nicht
zu bemerken. Wie ein Strom, unaufhaltsam, unbeeindruckt. Und mit einem Schlag
fallen all die Puzzleteile zu einem stimmigen Bild zusammen, ergeben alle meine
Beobachtungen einen Sinn. Das, was ich auf den ersten Blick als eine R�stung
abgetan hatte, ist vielmehr seine nat�rliche "`Haut"'. Das rote Leuchten
darunter? Sein "`Fleisch"'. Jedenfalls, wenn man ein Wesen aus fl�ssigem Stein
mit menschlichen Begriffen zu beschreiben versucht. Ich bl�ttere so schnell ich
kann vorsichtig zu den ersten Seiten des Buches vor mir zur�ck. Zu den Bildern,
die ich f�r die Darstellung verschiedener Felsformationen gehalten habe. Bei
erneuter, genauerer Betrachtung sind deutliche �hnlichkeiten mit dem gut zwei
Meter gro�en und fast genauso breiten wandelnden Berg dort am Kamin zu
erkennen. Unterdessen greift eben dieser Berg mit drei rot gl�henden Tentakeln,
die seinem rechten Arm entspringen, einen der Quader vor dem Kamin. Instinktiv
nehme ich an, dass der Klotz nun ebenfalls schmelzen w�rde, doch ich werde
entt�uscht. Obwohl "`entt�uscht"' irgendwie nicht das richtige Wort ist. Denn im
Griff der pulsierenden, wabernden Tentakel erhebt sich dieser Stein, den wohl
vier M�nner nicht gemeinsam h�tten heben k�nnen, in die Luft und folgt seinem
Tr�ger durch den Raum.

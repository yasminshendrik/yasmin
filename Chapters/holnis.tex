\section{Holnis}
Die F�hre wird erst in den Morgenstunden des n�chsten Tages zur�ck erwartet,
teilt man mir mit. Aber ich k�nne gerne eine Koje im F�hrhaus f�r die Nacht
buchen, dann w�re sicher gestellt, dass ich die Abfahrt nicht verpasse. Ein
Angebot, dass ich angesichts des unversch�mten Preises eher widerwillig annehme.

Ein schmaler Pfad f�hrt entlang der Steilk�ste weg vom F�hrhaus in Richtung des
endg�ltigen Endes der Landzunge. An der anderen Uferkante, die nicht ganz so
steil abf�llt, stehen noch zwei weitere kleine H�fe direkt am Wasser. In den
Fenstern brennt bereits Licht, aus den Schornsteinen steigt d�nner Rauch in den
sp�tnachmitt�glichen Himmel.

Das Wetter hat sich nur geringf�gig gebessert. Die Sonne kommt nur schwer durch,
aber der Wind weht nur schwach, sodass ihre Kraft ausreicht, der Luft eine
angenehme, wenn auch jahreszeitbedingt k�hle Temperatur zu verleihen.

An der Spitze der Landzunge windet sich der Weg langsam hinab zum Wasser. Sanft
pl�tschern die Wellen an den aus gro�en und kleinen Felsbrocken bestehenden
Strand. Der typische Geruch nach angesp�ltem Tang und toten Muscheln mischt sich
mit dem Geruch der B�ume und Str�ucher, die hier unmittelbar am Hang zu beiden
Seiten des Weges wachsen. Am Horizont sehe jede Menge Land, Inseln, vielleicht
auch Festland, ich wei� es nicht. Ein paar Fischerd�rfer lassen sich in der
diesigen Luft ausmachen. Wo wohl die F�hre anlanden wird? In der sich st�ndig
ver�ndernden Wasseroberfl�che fallen mir ein paar unbewegliche wei�e und
schwarze Punkte auf. M�wen und Kormorane, tippe ich. Also muss dieses Wasser mit
dem Meer verbunden sein. Eine Bucht? Oder vielleicht doch Inseln. Da ich nichts
besseres zu tun habe, setze ich mich hin und warte darauf, dass die Sonne hinter
mir untergeht. Dann kehre ich zum F�hrhaus zur�ck, um mein Lager f�r die Nacht
in Anspruch zu nehmen.
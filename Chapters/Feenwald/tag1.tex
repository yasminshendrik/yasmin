\chapter{Kleine Freunde}
\section{Tag 1}
Ich �ffne die Augen. Es ist so dunkel, dass ich nichts erkennen kann. Ich
versuche, mich zu bewegen und stelle fest, dass ich in einem Bett liege. Wie bin
ich hierher gekommen? "`Hallo?"', frage ich vorsichtig in die Dunkelheit. "`Ist
dort jemand? Hallo!"', versuche ich es nochmal lauter. "`Ihr seid also endlich
erwacht,"' antwortet eine helle, freundliche Stimme. "`Wo bin ich? Wer seid
Ihr?"'

Aus der Dunkelheit kommen zwei kleine Lichtpunkte auf mich zu. Zwischen ihnen
schwebt eine Schale. Als sie n�her kommen erkenne ich, dass die Lichtpunkte in
Wirklichkeit zwei kleine fliegende Wesen sind, die das Licht ausstrahlen und
zwischen sich die Schale tragen, obwohl diese f�r ihre zierlichen Gestalten viel
zu schwer zu sein scheint. "`Trinkt, damit Ihr wieder zu Kr�ften kommt!"' ert�nt
erneut diese Stimme, aus dem Mund der linken Figur. Vorsichtig setze ich mich
auf und nehme ihnen die Schale ab. Sie ist wirklich schwer, aus Ton geformt und
randvoll mit einer Fl�ssigkeit, die ich in dem schummrigen Licht der beiden
nicht n�her erkennen kann. Doch schlie߭lich gebe ich dem Grummeln in der
Magengegend nach und probiere einen Schluck. Es schmeckt sehr gut, ich kann
geradezu f�hlen, wie neues Leben meinen K�rper durchstr�mt.

Das erinnert mich an \Yasmin, "`Wo ist ..."', setze ich zu einer Frage an. Noch
bevor ich zuende sprechen kann sagt sie: "`Sprecht noch nicht. Schont Eure
Kr�fte, ruht Euch aus. Anschlie�end werden wir Eure Fragen beantworten. Sie wird
leben,"' beantwortet sie meine unausgesprochene Frage nach einer kurzen Pause.
Langsam trinke ich die Schale leer. So gut es mir auch geht, merke ich nun doch,
wie m�de ich immernoch bin. "`Danke!"', sage ich noch, bevor ich mich umdrehe
und die Augen wieder schlie�e.